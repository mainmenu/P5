\section{Conclusion and perspective}

In this paper we have presented a cube shaped inverted pendulum prototype for one of the sides of the cube.

We have designed a possible solution for the cube shaped inverted pendulum control.
The simulations what we have conducted during this project have proven that the control method used in the current designe works.
For the control of the system we have implemented state space based control method.

In order to make the cube jump up we modelled physical model for the jump up sequence.
In this section of the project we have made sure that it is possible to make the cube lift off the surface where it is laying on one of its sides.
The lift off is made possible by storing energy on the momentum wheel and then rapidly bringing it to stop.
Due to the enargy transfer from the wheel to the body we can make the system jump on to its corner.
We have concluded that the speed necessary for the lift off to the point where the cube starts balancing is relativly small compared to the maximum speed of the motor proposed for the system.

The balancing simulations have shown us the maximum capabilities of the motor used to keep it balanced untill the certain angle of offset.
Aswell we can see that the system reaction times are low what proves the efficiency of the proposed model.

Overall we can see that the system works and during this project we have learned alot about the modelling and controlling dynamic systems.

In future perspective of this project we see that it is good to build the whole cube.
When the whole cube has been designed and built we see different uses for the system.
For example reaction wheel based movement can be used for interplanetary exploration on data collecting robots.
Because if the desired robot can move around different surfaces without having any external moving parts it is much more environment proof.
It can be used in harsh conditions.

If the project would be carried on in upcoming semesters then we propose to have a bigger team to work on it because of the extensive possibilities of the future uses.
