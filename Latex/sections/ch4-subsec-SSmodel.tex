\subsection{Nonlinear dynamics of the setup}

In order to design the nonlinear state space controller for our system we must determine the nonlinear dynamic equation for the setup.
The following equation is describing the system by the torques that are applied on the system while it is tilted and the reaction wheel tries to restabilize the body to the balancing point according to the body's acceleration.

\begin{equation}
\label{eq:Non-linear dynamics11}
\ddot{\Theta_{b}}(I_{b}+m_{w}l^{2})=(m_{b}l_{b}+m_{w}l)gsin(\Theta_{b})-T_{m}-C_{b}\dot{\Theta_{b}}+C_{w}\dot{\Theta_{w}}
\end{equation}
Here we have the same equation made to correspond to the angular acceleration of the body.
It is made to be according to the body because in our nonlinear control we have two inputs first one is according to the body.
\begin{equation}
\label{eq:Non-linear dynamics12}
\ddot{\Theta_{b}}=\frac{(m_{b}l_{b}+m_{w}l)gsin(\Theta_{b})-T_{m}-C_{b}\dot{\Theta_{b}}+C_{w}\dot{\Theta_{w}}}{I_{b}+m_{w}l^{2}}
\end{equation}
Second equation for the nonlinear dynamics of the system shows the dynamics of the body according to the wheels movement.
\begin{equation}
\label{eq:Non-linear dynamics21}
\ddot{\Theta_{w}}(I_{b}+m_{w}l^{2})=\frac{(I_{b}+I_{w}+m_{w}l^{2})(T_{m}-C_{w}\dot{\Theta_{w}})}{I_{w}}-(m_{b}l_{b}+m_{w}l)gsin(\Theta_{b})-C_{b}\dot{\Theta_{b}}
\end{equation}
And now we make the equation according to the $\ddot{\Theta_{w}}$
\begin{equation}
\label{eq:Non-linear dynamics22}
\ddot{\Theta_{w}}=\frac{(I_{b}+I_{w}+m_{w}l^{2})(T_{m}-C_{w}\dot{\Theta_{w}})}{I_{w}(I_{b}+m_{w}l^{2})}-\frac{(m_{b}l_{b}+m_{w}l)gsin(\Theta_{b})-C_{b}\dot{\Theta_{b}}}{(I_{b}+m_{w}l^{2})}
\end{equation}
Both of the equations describe the dynamics of the system but the difference is that first one has the background system as the body.
Second dynamic equation is where the wheel is taken as the dynamic background system.
If we add up those equations we get the body's dynamic equation according to nonmoving background system.














\subsection{State Space Model}


Since we have determined our nonlinear dynamic equations we can compose our state space model.



\begin{align*}
	\dot{x}(t) = Ax(t) + Bu(t)\\
	{y}(t) = Cx(t) + Du(t)
\end{align*}
The matrices A and B are properties of the system and are determined by the system structure and elements.
The output equation matrices C and D are determined by the particular choice of output variables\cite{state1}.
$dot{x}$ and $y$ are the input and output matrices.

We assigne the state vector $x$ as follows	

\[ x(t) =  \left[ \begin{array}{ccc}
\Theta_{b}  \\
\dot \Theta_{b}  \\
\dot \Theta_{w} \\\end{array} \right]\] 
where

$\Theta_{b}$ is the angular position of the body,$\dot{ \Theta_{b}}$ is the angular velocity of the body and $\dot{\Theta_{w}}$ is the angular velocity of the reaction wheel.


Our systems output vector is:

\[ y(t) =  \left[ \begin{array}{ccc}
\Theta_{b}  \end{array} \right]\] 

And the input vector is:

\[ u(t) =  \left[ \begin{array}{ccc}
i  \end{array} \right]\] 



For calculating the matrices of the linearized system around the  $x(0) = [0;0;0]$ point we are using the equations from the last section (\ref{eq:Non-linear dynamics12}) and (\ref{eq:Non-linear dynamics21}).

The property matrix A is given\cite{state1}


\[ A =  \left( \begin{array}{ccc}
a_{1,1} & a_{1,2} & a_{1,3} \\
a_{2,1} & a_{2,2} & a_{2,3} \\
a_{3,1} & a_{2,2} & a_{2,3} \\\end{array} \right)\] 
according to what we can give our system propery matrix A as:
\[ A =  \begin{bmatrix}
0 &  1 & 0 \\
\frac{(m_{b}I_{b}+(m_{w}l))g}{I_{b}+m_{w}l^2} & -\frac{C_{b}}{(I_{b}+m_{w}*l^2)} & \frac{C_{w}}{I_{b}+m_{w}l^2} \\
-\frac{(mbI_{b}+m_{w}l)g}{I_{b}+m_{w}l^2} & \frac{C_{b}}{I_{b}+m_{w}l^2} & -\frac{C_{w}(I_{b}+I_{w}+m_{w}l^2)}{I_{w}(I_{b}+m_{w}l^2)} \end{bmatrix} .\] 



Second property matrix B we get from
$3 \times 1$~matrix

\[ B =  \left( \begin{array}{ccc}
b_{1,1}  \\
b_{2,1}  \\
b_{3,1}  \\\end{array} \right)\] 
is according to the formula above\cite{state1}
\[ B = \begin{bmatrix}
        0  \\
-\frac{K_{m}}{I_{b}+m_{w}l^2}\\
\frac{K_{m}(I_{b}+I_{w}+m_{w}l^2)}{I_{w}(I_{b}+m_{w}l^2)}
       \end{bmatrix}\]



The output matrix C is represented as follows

\[ C =  \left( \begin{array}{ccc}
c_{1,1}  & c_{1,2}  & c_{1,3}  \\\end{array} \right)\] 
is given by the formula\cite{state1}
\[ C =\begin{bmatrix}
       1 & 0 & 0
      \end{bmatrix}
\] 

And finally the second output property matrix is represented as:


\[ D =  \left( 
d_{1,1}     \right)\] 
is given by the formula
\[ D= \begin{bmatrix}
    0
   \end{bmatrix}
\] 

``When we use the continuous time system given by $\dot{x}(t) = Ax(t) + Bu(t)$ and descretized using zero-order hold and the resulting discrete time model is given by
\begin{equation}
 x[k+1]=A_{d}[k]+B_{d}u[k], k\in \mathbb{N}
\end{equation}
where $A_{d}$ and $B_{d}$ are the discrete-time counterparts of the continous time state matrix $A$ and input matrix $B$.
For the sake of simplicity we use the same notation to represent the continuous and discrete time versions of the state $x$ and input $u$.

Using the above discrete time model, a Linear Quadratic Regulator (LQR) feedback controller of the form 
\begin{equation}
 u[k]=-K_{d}(\hat{\theta_{b}}[k],\hat{\dot{\theta_{b}}}[kk],\hat{\dot{\theta_{w}}}[k])
\end{equation}
was designed, where $K_{d}=(K_{d1},K_{d2},K_{d3})$ is the LQR feedback gain ''\cite{cubli12}

 
 
 
 
 
 
 
 
 
 

