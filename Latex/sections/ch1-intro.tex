\section{Introduction}
In this project we will try to understand and implement controlling of an unbalanced device via using non-linear control.
Our aim is to build a prototype for a cube what would be able to move and balance on its edges without external forces.
The cube would move and balance itself thanks to the embedded reaction wheel.
Goal of our project is to make one side of the cube and try to implement the nonlinear controller that we develop during this project.
The idea was adapted from an older work done in ETH Zurich, Switzerland by Mohanarajah Gajamohan, Michael Merz, Igor Thommen and Raffaello D’Andrea.\cite{cubli12}
In the mentioned project they finally completed the whole cubic, but in our case we are only looking forward to build one side of it in the duration of this project.

We have taken up this project because it is a challenge for us to make this kind of a controller that can balance the system and maintain it.
Similar kind of systems are used in self assemling robots and satellites.\cite{selfrobot}\cite{satellite}
Since the work on using the reaction wheel and other ideas in these kind on projects is still ongoing we feel that this work is actual and worth the effort.

The principle of the device is simple. 
First we make the wheel go very fast and then bring it to a rapid stop by breaking the wheel with a servo motor. 
Energy what was stored on the wheel then transforms over to the whole body and the system should move to the desired position.
If the modelling proves to be corret we should have the body in upright position.

Balancing of the body is done by using the torque of the reaction wheel dependent on the side that it is leaning towards to.
To measure the offset of the body we use the Earth's gravity and an accelerometer.
The angle of the offset and thus the required torque is calculated by the projection of the gravity vector what we get from the sensor.

Controlling of the system is done by a non-linear controller implemented by software on a microcontroller.
Rapid breaking of the wheel is done by a servo motor mounted on the body.
In next chapters we will give more in depth about what we did and how.
